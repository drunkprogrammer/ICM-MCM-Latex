\documentclass{mcmthesis}
\mcmsetup{CTeX = false,   % 使用 CTeX 套装时,设置为 true
        tcn = 74911, problem = D,
        sheet = true, titleinsheet = true, keywordsinsheet = true,
        titlepage = true, abstract = true}
\usepackage{palatino}
\usepackage{lipsum}
\usepackage{indentfirst}
\title{Analysis and Optimization to the Process of Airport Security Check}%小标题
\author{\small Team \#74911}
\date{\today}
\begin{document}
\begin{abstract}
\par Security check is essential to the safety of airline passengers. However, it is also a time
consuming process. Passengers have to wait in lines, get ID check, prepare the belongings and get screening inspection and sometimes it even takes hours of time to finish.
Besides, the variance of the waiting time in the checkpoints line is also very high, which
is difficult for passengers to decide when to reach the airport. Therefore, we are required
to figure out the potential issues and make some modifications to the current process
that can reduce the average waiting time and its variance.
\par We develop the \textbf{basic model} to simulate the current situation of American airports'
security check and study the properties of Pre-Check policy. Though more `Pre-Check'
passengers can help alleviate congestion to some degree, the time in the process of wait
for screening is the overwhelming majority of the time spent in security check. It occupy
over 98\% of total time during rush hours. In addition, we also plot the distribution of
different total waiting time, and find that it is nearly a uniform distribution over several
hours. The variance is hence too high for passengers to predict their total waiting time.
\par Based on this conclusion, we mainly build two efficient modification models to optimize the current situation. The first one is \textbf{`multi-passenger linkage model of belongings preparation (MPL)'}. It shows that if the airport opens up a small zone for several
passengers to prepare for screening at the same time, the average time and variance will
obviously decrease. The other one is \textbf{`priority-based queuing model (PBQ)'}. It suggests
airports to set some appropriate time interval limits to provide passengers who is to take
off with great convenience without increasing the average time. Besides, it can encourage people not to arrive at airports so early that aggravate congestion.
\par We also speculate about the TSA's latest institution for the elderly and children, but
it does not make much contribution. At last, we consider the Chinese cultural norms
and apply our model to Chinese airports. We construct the \textbf{cut-in-line} model. The result
shows that MPL and PBQ models can still keep their high effectiveness. In other word,
our models have high stability, high error-tolerant rate and extensive applicability.
\begin{keywords}
Security Check; Poission Distribution; Gaussian Distribution; Queuing Methods
\end{keywords}
\end{abstract}
\maketitle

\newpage
\tableofcontents
\newpage

\section{Restatement of the Problem}
\subsection{Background}
\subsubsection{An Overview of the Problem}
\par In recent years, as the threat of terrorism is upgrading, American society has attached
great significance to the security of the airport. A key part of the security measures at
the airport is security checkpoints, where the passengers and their belongings should be
inspected for dangerous objects. However, if the passengers takes a long time to pass the
security checkpoints, there will be a negative impact on the customers' feelings, then the
interests of the airlines can be influenced.
\par There are two major problems in Chicago's O'Hare international airport and other
American airports. First, it is annoying that there are sometimes over-long queues at
the airports, but we cannot give an explanation or a prediction. Second, the variance of
waiting time is high, that is to say, it is difficult for the travelers to find a suitable time
to reach the airport without being unnecessarily early or missing the plane. Therefore it
makes great sense to solve the congestion problem.

\subsubsection{The Process of Security Check}
The process of security check can be decomposed into following processes:
\begin{itemize}
\item \textbf{Arriving process}:Passengers are assigned at the beginning of our model.
\item \textbf{Process of waiting at the checkpoints}: Passengers wait at the checkpoints in Zone
A and then move to the subsequent queue in Zone B.
\item \textbf{ID check process}: Passengers get the identification check in this process.
\item \textbf{Process of waiting for screening}: Passengers wait in lanes for screening.
\item \textbf{Preparing process}: After reaching the front of the line, passengers place their belongings, including their jackets, liquid containers, computers and so on.
\item \textbf{Screening process}: Passengers get screened by the millimeter scanner while the
goods are scanned by the X-ray, then they retrieve their belongings.
\item \textbf{Additional inspection process}: If the passengers fail in the screening process, they
should get additional inspections in Zone D.
\end{itemize}

\subsection{literature review}
To simplify the problem and facilitate the implementation of computer language,
researchers treat time as a discrete variable in the analysis of a single-server queuing
system at first[1]. After that, some people have applied Queuing Theory to the dynamic
allocation of airport security resources, which used pure Markov chain, discontinuous
Markov process, Poisson process, statistical properties of transition probability equation
and the birth and death process[2] [3]. What's more, some other people try to achieve a goal of virtual queuing instead of real one in order to get a smoother distribution of arriving passengers[4]. We expect to find some convenient and intuitive modifications for
the airport security based on the previous wisdom and some necessary math knowledge
like Queuing Theory, Probability and Statistics[5].


\subsection{The Task at Hand}
\begin{itemize}
\item Construct a mathematical model to simulate the situation of the current security
checkpoints and find an explanation for the phenomenon of congestion.
\item Propose some modifications to improve the current process.
\item Take the various behaviors of travelers into consideration, then discuss the influence of those factors on the model.
\item On the basis of the model, give some policy and procedural suggestions to ameliorate the situation of the security checkpoints.
\end{itemize}
%\emph{center of percussion} [Brody 1986], 

%\begin{Theorem} \label{thm:latex}
%\LaTeX
%\end{Theorem}
%\begin{Lemma} \label{thm:tex}
%\TeX .
%\end{Lemma}
%\begin{proof}
%The proof of theorem.
%\end{proof}






\section{Model Assumptions and Notations}
\subsection{Assumptions and Justifications}
In order to simplify the course of modeling and draw some reasonable conclusions
from our model, we make assumptions as follows:

\begin{itemize}
\item The excel data can embody the overall level of the airport to some degree. We use
the differential method to calculate out the mean value and the standard deviation
of the time consumed in different processes.
\item  The intervals of the passengers' arrivals obey Poisson distribution. Because the
time when a passenger arrives is unrelated to that of the last passenger, therefore
we assume that the intervals of the passengers' arrivals obey Poisson distribution.
The mean values of the distribution are obtained from the excel data.
\item The time of ID check and millimeter wave scanning obeys Gaussian distribution.
In this two processes, we suppose that the staff of the airport are skillful, that is
to say the time cost in these processes fluctuates around a mean value. Thus we
assume that they obey Gaussian distribution. The mean values and the standard
deviations are also calculated out according to the excel data.
\item  The time interval of two examinations is 11s to pass the X-ray scanning and the
time interval between two placements is 2s. According to the excel data, we can see
that the interval is sometimes around 2s and sometimes around 11s. We interpret
it as: in one time of examination, the inspector inspects several pieces of luggage.
There are time intervals between each examinations.
\item All of the passengers know the rules of the airport. They will go to the back of the
shortest queue after arrival.
\item There are no emergency incidents happened in the airport. What we discuss is the
airport in order and it is clear that any emergency incident can break the order of
any airport.
\item Pre-Check passengers can choose to go through the regular passage of security
check when the Pre-Check passage is crowded. As for Pre-Check passengers, it
takes them the same time either in Pre-Check passages or in regular passages.
\item The time of preparing the belongings is 15 $\pm$ 5 + 5X s in the Pre-Check passage,
while it is 45 $\pm$ 10+ 5X s in the ordinary passage. This is based on our experiments.
X is the number of the pieces of luggage.
\item In our model, there are 4 ID checkpoints in Zone A and 4 checking lanes in Zone B.
\item We do not take the transiting time between different stations into consideration.

\end{itemize}



\subsection{Notations}
\par Here we list the symbols and notations used in this paper, as shown in Table 1. Some
of them will be defined later in the following sections.


\begin{table}
\begin{center}
\caption{Notations}
\begin{tabular}{cl}
\toprule
Symbol & Description \\%表格标题
\midrule
$t_\alpha$ & the time interval between two passengers' arrivals\\
$t_p$ & the time interval between two Pre-Check passengers' arrivals \\
$t_r$ & the time interval between two regular passengers' arrivals\\
$t_{waitA}$ & the time of waiting at Zone A\\
$t_{waitB}$ & the time of waiting at Zone B\\
$t_{check}$ & the time of identification check\\
$\sigma _{check}$ & the standard deviation of $t_{check}$\\
$t_{xray}$ & the time of X-ray scanning \\
$t_{milli}$ & the time of millimeter wave scanning\\
$t_{pre}$ & the time of preparing\\
$\sigma _{pre}$ &the time of preparing \\
$t_{screening}$ & the standard deviation of $t_{type}$\\
$T$ &the total time of the security check of a single person \\
$T_{avg}$ & the total time of the security check process of a group of people\\
$\sigma_{T}$ & the standard deviation of $T_{avg}$\\
$T_{r}$ & the total time consumed by regular passengers\\
$T_{p}$ & the total time consumed by PreCheck passengers\\
$\sigma_{Tr}$ & the standard deviation of $T_{r}$\\
$\sigma_{Tp}$& the standard deviation of $T_{p}$\\
$N_{prepare}$ &the standard deviation of $T_{r}$ \\
$R_{passenger}$ & the number of passengers preparing their belongings\\
$R_{lane}$ & the ratio of $t_{p}$ to $t_{r}$ \\
$R_{oyp}$ & the ratio of Pre-Check lanes to regular lanes \\
$R_{cut}$ & the percentage of people over 70 or under 12\\
\bottomrule
\end{tabular}
\end{center}
\end{table}


\section{The Basic Model of Security Checkpoints}
\par In this section, we develop the basic model to explain the current situation of American airports. The basic model simulates the processes mentioned in Section 1.1.2. Then
we discover that the process of waiting for screening is the most time consuming. The
Pre-Check policy of TSA is effective, but we need to find more strategies to solve the
problem.
\subsection{The Design of the Model}
\par In order to demonstrate our basic model clearly, we divide it into the four following
sub models.
\begin{itemize}
\item Inflow model: This model is designed to simulate the random arrivals of the passengers.
\item Sub model two: Attributes generating model: In this model, we attribute different
properties to the passengers that will influence the time length of security check.
\item Queuing model: This model studies the process of waiting in lines.
\item Screening model: The preparing process and the screening process are discussed
in this model.
\end{itemize}
\subsection{Sub Models}
\subsubsection{Inflow Model}
\par The inflow model simulates the stochastic arrivals of passengers at the entrance of
the airport. According to the excel data, we can obtain that the average time interval of
passengers' arrivals. We can assume that the arrival of each passenger obeys Poisson
distribution in Equation \eqref{aa}. $ \lambda $ is the average time interval between the arrivals of two
passengers according to the excel data, and N is the number of seconds between the
arrivals of two passengers in our stochastic process.
\begin{equation}
P(t = N) = {e^{ - \lambda }}\frac{{{\lambda ^N}}}{{N!}} \label{aa}
\end{equation}

\subsubsection{Attributes Generation Model}
After a passenger is assigned in the program, we give some relevant attributes to
the passengers, including the time for ID check, the time for millimeter wave scanning,
the pieces of luggage they carry, the time for preparing the personal belongings and
whether the passenger has enrolled in 'Pre-Check'. We assume that the first two factors
obey Gaussian distribution in Equation (2). Then we calculate the mean value and the
standard deviation of the first three affecting factors. Those of the fourth one are given
in the assumptions.

\begin{figure}[h]
\small
\centering
\includegraphics[width=12cm]{mcmthesis-aaa.eps}
\caption{aa} \label{fig:aa}
\end{figure}

%\lipsum[8] \eqref{ab}
%\begin{equation}
%a^2 \label{ab}
%\end{equation}

\[
  \begin{pmatrix}{*{20}c}
  {a_{11} } & {a_{12} } & {a_{13} }  \\
  {a_{21} } & {a_{22} } & {a_{23} }  \\
  {a_{31} } & {a_{32} } & {a_{33} }  \\
  \end{pmatrix}
  = \frac{{Opposite}}{{Hypotenuse}}\cos ^{ - 1} \theta \arcsin \theta
\]
\lipsum[9]

\[
  p_{j}=\begin{cases} 0,&\text{if $j$ is odd}\\
  r!\,(-1)^{j/2},&\text{if $j$ is even}
  \end{cases}
\]

\lipsum[10]

\[
  \arcsin \theta  =
  \mathop{{\int\!\!\!\!\!\int\!\!\!\!\!\int}\mkern-31.2mu
  \bigodot}\limits_\varphi
  {\mathop {\lim }\limits_{x \to \infty } \frac{{n!}}{{r!\left( {n - r}
  \right)!}}} \eqno (1)
\]

\section{Calculating and Simplifying the Model  }
\lipsum[11]

\section{The Model Results}
\lipsum[6]

\section{Validating the Model}
\lipsum[9]

\section{Conclusions}
\subsection{Strength and Weakness of Our Model}
\subsection{Future Application of Models}

\begin{itemize}
\item \textbf{Applies widely}\\
This  system can be used for many types of airplanes, and it also
solves the interference during  the procedure of the boarding
airplane,as described above we can get to the  optimization
boarding time.We also know that all the service is automate.
\item \textbf{Improve the quality of the airport service}\\
Balancing the cost of the cost and the benefit, it will bring in
more convenient  for airport and passengers.It also saves many
human resources for the airline. \item \textbf{}
\end{itemize}

\begin{itemize}
\item \textbf{Applies widely}\\
This  system can be used for many types of airplanes, and it also
solves the interference during  the procedure of the boarding
airplane,as described above we can get to the  optimization
boarding time.We also know that all the service is automate.
\item \textbf{Improve the quality of the airport service}\\
Balancing the cost of the cost and the benefit, it will bring in
more convenient  for airport and passengers.It also saves many
human resources for the airline. \item \textbf{}
\end{itemize}

\begin{thebibliography}{99}
\bibitem{1} D.~E. KNUTH   The \TeX{}book  the American
Mathematical Society and Addison-Wesley
Publishing Company , 1984-1986.
\bibitem{2}Lamport, Leslie,  \LaTeX{}: `` A Document Preparation System '',
Addison-Wesley Publishing Company, 1986.
\bibitem{3}\url{http://www.latexstudio.net/}
\bibitem{4}\url{http://www.chinatex.org/}
\end{thebibliography}

\begin{appendices}

\section{First appendix}

\lipsum[13]

Here are simulation programmes we used in our model as follow.\\

\textbf{\textcolor[rgb]{0.98,0.00,0.00}{Input matlab source:}}
\lstinputlisting[language=Matlab]{./code/mcmthesis-matlab1.m}

\section{Second appendix}

some more text \textcolor[rgb]{0.98,0.00,0.00}{\textbf{Input C++ source:}}
\lstinputlisting[language=C++]{./code/mcmthesis-sudoku.cpp}

\end{appendices}
\end{document}

%% 
%% This work consists of these files mcmthesis.dtx,
%%                                   figures/ and
%%                                   code/,
%% and the derived files             mcmthesis.cls,
%%                                   mcmthesis-demo.tex,
%%                                   README,
%%                                   LICENSE,
%%                                   mcmthesis.pdf and
%%                                   mcmthesis-demo.pdf.
%%
%% End of file `mcmthesis-demo.tex'.
